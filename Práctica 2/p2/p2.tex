\documentclass[]{article}

\usepackage[left=2.00cm, right=2.00cm, top=2.00cm, bottom=2.00cm]{geometry}
\usepackage[spanish,es-noshorthands]{babel}
\usepackage[utf8]{inputenc} % para tildes y ñ
\usepackage{graphicx} % para las figuras
\usepackage{xcolor}
\usepackage{listings} % para el código fuente en c++

\lstdefinestyle{customc}{
  belowcaptionskip=1\baselineskip,
  breaklines=true,
  frame=single,
  xleftmargin=\parindent,
  language=C++,
  showstringspaces=false,
  basicstyle=\footnotesize\ttfamily,
  keywordstyle=\bfseries\color{green!40!black},
  commentstyle=\itshape\color{gray!40!gray},
  identifierstyle=\color{black},
  stringstyle=\color{orange},
}
\lstset{style=customc}


%opening
\title{Práctica 2. Programación dinámica}
\author{\input{../autor}}


\begin{document}

\maketitle

%\begin{abstract}
%\end{abstract}

% Ejemplo de ecuación a trozos
%
%$f(i,j)=\left\{ 
%  \begin{array}{lcr}
%      i + j & si & i < j \\ % caso 1
%      i + 7 & si & i = 1 \\ % caso 2
%      2 & si & i \geq j     % caso 3
%  \end{array}
%\right.$

\begin{enumerate}
\item Formalice a continuación y describa la función que asigna un determinado valor a cada uno de los tipos de defensas.

Como estructuras necesarias utilizo 2 vectores para almacenar los nodos AStarNode* usados por el algoritmo A*, uno llamado 'abiertos' que contiene los que aun están por visitar y ser procesados, y el otro 'cerrados' con los que ya han sido visitados. También utilizamos otro nodo AStarNode* llamado current que representa al nodo actual y por último hacemos uso del montículo de la STL (make\_heap) para reordenar el vector abiertos cada vez que se produzca una actualización en los parámetros de un nodo current.

El funcionamiento del algoritmo sigue la estructura del algoritmo A*: el nodo current es expandido y miramos sus hijos, los cuales se irán guardando en el vector de abiertos para poder visitarlos si no lo estuvieran. Los nodos que estén en este último vector se usarán para comprobar si es más conveniente ir al nodo objetivo mediante el padre del nodo actual(current) o a través del propio current. Una vez un nodo ha sido visitado se guarda en el vector de cerrados para no volver a visitarlo.

Para calcular que camino es el mejor, utilizamos un coste adicional a cada celda que consiste en calcular la distancia euclídea desde cada una de ellas hasta la celda donde esta situada la defensa principal, es decir, cuanto más lejos esté, mayor valor tendrá. Para ésto hacemos uso de una matriz de float llamada additionalCost. 

Finalmente, una vez hemos obtenido la solución, devolvemos el camino almacenándolo en la lista de Vector3 llamada 'path'. 

\item Describa la estructura o estructuras necesarias para representar la tabla de subproblemas resueltos.

\begin{verbatim}
using namespace Asedio;

Vector3 cellCenterToPosition(int i, int j, float cellWidth, float cellHeight){ 
    return Vector3((j * cellWidth) + cellWidth * 0.5f, (i * cellHeight) + cellHeight * 0.5f, 0); 
}

void DEF_LIB_EXPORTED calculateAdditionalCost(float** additionalCost
                   , int cellsWidth, int cellsHeight, float mapWidth, float mapHeight
                   , List<Object*> obstacles, List<Defense*> defenses) {

    float anchoCelda = mapWidth / cellsWidth;
    float altoCelda = mapHeight / cellsHeight;

    for(int i = 0 ; i < cellsHeight ; ++i) {
        for(int j = 0 ; j < cellsWidth ; ++j) {

            Vector3 cellPosition = cellCenterToPosition(i, j, anchoCelda, altoCelda);
            
            //Para el coste sumamos la distancia euclidea existente entre la celda donde
             esta situada la defensa principal y la celda actual.
            std::list<Defense*>::iterator it = defenses.begin();
            additionalCost[i][j] = _distance((*it)->position, cellPosition);
        }
    }  
}  

bool comparar(AStarNode* i, AStarNode* j){
	return (i->F > j->F);
}

void DEF_LIB_EXPORTED calculatePath(AStarNode* originNode, AStarNode* targetNode
                   , int cellsWidth, int cellsHeight, float mapWidth, float mapHeight
                   , float** additionalCost, std::list<Vector3> &path) {

    bool encontrado = false;
    float anchoCelda = mapWidth/cellsWidth;
    float altoCelda = mapHeight/cellsHeight;

    //Aplicamos algoritmo A*:
 
    AStarNode* current = originNode;
    std::vector<AStarNode*> abiertos;   
    std::vector<AStarNode*> cerrados;

    current->H = _sdistance(current->position, targetNode->position);   //Distancia estimada
     entre el nodo actual(origen) y el objetivo
    current->F = current->G + current->H;
    abiertos.push_back(current);
    std::make_heap(abiertos.begin(), abiertos.end(), comparar);

    while(encontrado == false && abiertos.size() > 0){
    	current = abiertos.front();
    	std::pop_heap(abiertos.begin(), abiertos.end(), comparar);
    	abiertos.pop_back();
    	cerrados.push_back(current);

    	if(current == targetNode)
    		encontrado = true;
    	else{ 
    		std::list<AStarNode*>::iterator it = current->adjacents.begin();
    		for(it; it != current->adjacents.end(); it++){
    			if(cerrados.end() == std::find(cerrados.begin(), cerrados.end(), (*it))){
    				if(abiertos.end() == std::find(abiertos.begin(), abiertos.end(), (*it))){
    					int posX = (*it)->position.x/anchoCelda;
    					int posY = (*it)->position.y/altoCelda;
    					(*it)->parent = current;
    					(*it)->G = current->G + _distance(current->position, (*it)->position)
    					 + additionalCost[posX][posY];
    					(*it)->H = _sdistance((*it)->position, targetNode->position);
    					(*it)->F = (*it)->G + (*it)->H;
    					abiertos.push_back((*it));
    					std::make_heap(abiertos.begin(), abiertos.end(), comparar);
    				}
    				else{  
    					float distancia = _distance(current->position, (*it)->position);
    					if((*it)->G > current->G + distancia){
    						(*it)->parent = current;
    						(*it)->G = current->G + distancia;
    						(*it)->F = (*it)->G + (*it)->H;
    						std::sort_heap(abiertos.begin(), abiertos.end(), comparar);
    					}
    				}
    			}
    		}
    	}
    }

    //Ahora recuperamos el camino a seguir por el UCO:

    current = targetNode;
    path.push_front(current->position);

    while(current->parent != originNode){
        current = current->parent;
        path.push_front(current->position);
    }
}
\end{verbatim}

\item En base a los dos ejercicios anteriores, diseñe un algoritmo que determine el máximo beneficio posible a obtener dada una combinación de defensas y \emph{ases} disponibles. Muestre a continuación el código relevante.

\begin{lstlisting}
void mochila(std::list<Defense*> defenses, float** &TSP, unsigned int ases, float* valores, unsigned int* costes){

	for(int j=0; j<ases; j++){	
		if(j < costes[0])
			TSP[0][j] = 0;
		else
			TSP[0][j] = valores[0];
	}

	for(int i=1; i<defenses.size(); i++){	
		for(int j=0; j<ases; j++){
			if(j < costes[i])
				TSP[i][j] = TSP[i-1][j];
			else
				TSP[i][j] = std::max(TSP[i-1][j], TSP[i-1][j-costes[i]]+valores[i]);
		}
	}
}
\end{lstlisting}

\item Diseñe un algoritmo que recupere la combinación óptima de defensas a partir del contenido de la tabla de subproblemas resueltos. Muestre a continuación el código relevante.

Conjunto de candidatos: las celdas del mapa.

Un conjunto de candidatos seleccionados: celdas seleccionadas.

Función solución: comprobamos que la defensa no haya sido colocada, e iteramos las celdas candidatas hasta que dicha defensa sea colocada, poniendo a true nuestra variable booleana.

Función de selección: recorremos nuestra matriz de celdas buscando siempre la celda con mayor puntuación, así nos aseguramos de escoger siempre el candidatos más idóneo para nuesra solución.

Función de factibilidad: si la celda seleccionada cumple los requisitos explicados anteriormente en el ejercicio 2, entonces es factible.

Función objetivo: colocar las defensas.

Objetivo: defender la defensa principal el mayor tiempo posible antes de su destrucción por los ucos.

\end{enumerate}

Todo el material incluido en esta memoria y en los ficheros asociados es de mi autoría o ha sido facilitado por los profesores de la asignatura. Haciendo entrega de este documento confirmo que he leído la normativa de la asignatura, incluido el punto que respecta al uso de material no original.

\end{document}
