Conjunto de candidatos: las celdas del mapa.

Un conjunto de candidatos seleccionados: celdas seleccionadas.

Función solución: comprobamos que la defensa no haya sido colocada, e iteramos las celdas candidatas hasta que dicha defensa sea colocada, poniendo a true nuestra variable booleana.

Función de selección: recorremos nuestra matriz de celdas buscando siempre la celda con mayor puntuación, así nos aseguramos de escoger siempre el candidatos más idóneo para nuesra solución.

Función de factibilidad: si la celda seleccionada cumple los requisitos explicados anteriormente en el ejercicio 2, entonces es factible.

Función objetivo: colocar las defensas.

Objetivo: defender la defensa principal el mayor tiempo posible antes de su destrucción por los ucos.